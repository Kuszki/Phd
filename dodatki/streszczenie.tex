\chapter*{Streszczenie}\addcontentsline{toc}{chapter}{Streszczenie}

W pracy przedstawiono metodę wyznaczania wartości niepewności rozszerzonych wielkości wyjściowych algorytmów dyskretnej transformacji falkowej, uniwersalną dla dowolnej kombinacji parametrów tych algorytmów. Praca przedstawia podstawowe informacje dotyczące algorytmów transformacji falkowej, wymienia obszary i przykłady aplikacji omawianych algorytmów, a także przedstawia macierzowy sposób zapisu tych algorytmów. Aby umożliwić analizę właściwości metrologicznych torów pomiarowych zawierających w swojej strukturze algorytmy transformacji falkowej, zaproponowano model błędów opisujący kolejne elementy toru pomiarowego, uwzględniający wzajemne relacje zachodzące pomiędzy jego fragmentami. Proponowany model błędów pozwala określić zależności pomiędzy kolejnymi sygnałami błędów zawartymi w sygnale pomiarowym oraz umożliwia oszacowanie parametrów wypadkowego sygnału błędu na wyjściu algorytmu. Zaproponowany model błędów uwzględnia również widmo przetwarzanego sygnału w ocenie niedokładności wielkości wyjściowych toru pomiarowego. Na potrzeby pracy wprowadzono podział na statyczne, dynamiczne i losowe sygnały błędów oraz podział rozważający genezę analizowanego sygnału, wyróżniający sygnały błędów własnych i propagowanych. Wszystkie przedstawione w pracy zależności zostały zweryfikowane symulacyjnie za pomocą metody Monte-Carlo oraz pomiarowo, stosując w tym celu zbudowany na potrzeby pracy tor pomiarowy. Praca przedstawia przykłady aplikacji zaproponowanej metody, w których uwzględniono typowe scenariusze wraz z przedstawieniem różnych możliwości rozwiązań pod kątem przyjętych uproszczeń i ich konsekwencji dla dokładności uzyskiwanych wyników.
