\chapter*{Streszczenie}\addcontentsline{toc}{chapter}{Streszczenie}

W pracy przedstawiono metodę wyznaczania niepewności wielkości wyjściowych algorytmów dyskretnej transformacji falkowej, uniwersalną dla dowolnej kombinacji parametrów tego algorytmu. Praca przedstawia podstawowe informacje dotyczące algorytmów transformacji falkowej, wymienia obszary i przykłady aplikacji omawianych algorytmów, a także przedstawia macierzowy sposób zapisu algorytmu transformacji falkowej. Aby umożliwić analizę metrologiczną omawianych algorytmów, zaproponowano model błędu umożliwiający opis właściwości metrologicznych kolejnych elementów toru pomiarowego oraz opis właściwości toru składającego się z kaskadowego połączenia opisywanych elementów, który przedstawia zależności pomiędzy kolejnymi błędami zawartymi w sygnale pomiarowym oraz umożliwia oszacowanie parametrów sygnału błędu na wyjściu algorytmu. Zaproponowany model błędu uwzględnia widmo sygnału w ocenie niepewności wielkości wyjściowej całości toru pomiarowego i wprowadza podział na statyczne, dynamiczne i losowe błędy w przetwarzanym sygnale. Wszystkie przedstawione w pracy zależności są weryfikowane za pomocą metody Monte-Carlo oraz eksperymentalnie stosując zbudowany na potrzeby pracy tor pomiarowy. Praca przedstawia przykłady aplikacji zaproponowanej metody, w których uwzględniono typowe scenariusze wraz z przedstawieniem różnych możliwości rozwiązań pod kątem przyjętych uproszczeń.
