\chapter*{Streszczenie}\addcontentsline{toc}{chapter}{Streszczenie}

W pracy przedstawiono metodę wyznaczania wartości niepewności rozszerzonych wielkości wyjściowych torów pomiarowych zawierających w swojej strukturze algorytmy transformacji falkowej. Przedstawiona metoda jest uniwersalna dla dowolnych algorytmów transformacji falkowej, które przetwarzają dane z dziedziny liczb rzeczywistych, niezależnie od stosowanych parametrów algorytmu. Przedstawiony w pracy model błędów, opisujący właściwości metrologiczne toru pomiarowego, umożliwia opis deterministycznych i niedeterministycznych sygnałów błędów oraz uwzględnia widmo przetwarzanego przez tor pomiarowy sygnału w ocenie jego właściwości. Na potrzeby pracy wprowadzono podział na statyczne, dynamiczne i losowe sygnały błędów oraz podział rozważający genezę analizowanego sygnału, wyróżniający sygnały błędów własnych i propagowanych. W pracy przedstawiono, w jaki sposób algorytmy transformacji falkowej przetwarzają obecne w sygnale wejściowym sygnały błędów oraz przedstawiono ich rolę we wprowadzaniu sygnałów błędów własnych. Aplikacja zaproponowanej metody wyznaczania wartości wypadkowej niepewności rozszerzonej jest możliwa w czasie rzeczywistym, również w przypadku zmiany parametrów związanych z modelem błędów analizowanego toru pomiarowego i nie wymaga w tym celu stosowania metody Monte-Carlo. Poza rozważaniami teoretycznymi, praca przedstawia przykład aplikacji zaproponowanej metody analizy, odpowiedni dla przypadku gdy projektant toru pomiarowego stosuje gotową implementację algorytmu transformacji falkowej i nie posiada eksperckiej wiedzy na temat działania tych algorytmów. Wszystkie przedstawione w pracy zależności zostały zweryfikowane symulacyjnie, za pomocą metody Monte-Carlo, oraz pomiarowo, stosując w tym celu zbudowany na potrzeby pracy tor pomiarowy. Praca poświęca najwięcej uwagi algorytmom dyskretnej transformacji falkowej, natomiast stosowanie zaproponowanej metody analizy jest możliwe również w przypadku pozostałych odmian algorytmu.
