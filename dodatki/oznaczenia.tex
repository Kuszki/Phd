\chapter*{Wykaz oznaczeń}\addcontentsline{toc}{chapter}{Wykaz oznaczeń}

\begin{longtable}[l]{ l @{~~--~~} p{376pt} }
$\omega$                        & pulsacja sygnału, gdzie $\omega = 2\pi f$ \\
$f$                             & częstotliwość sygnału, gdzie $f = \frac{\omega}{2\pi}$ \\
$T_{p}$                         & okres próbkowania sygnału, gdzie $T_{p} = \frac{1}{f_{p}}$ \\
$\omega_{p}$                    & pulsacja próbkowania sygnału, gdzie $\omega_{p} = 2\pi f_{p}$ \\
$f_{p}$                         & częstotliwość próbkowania sygnału, gdzie $f_{p} = \frac{\omega_{p}}{2\pi}$ \\
$\omega_{n}$                    & pulsacja znormalizowana, gdzie $\omega_{n} = \omega T_{p}$ \\
$f_{n}$                         & częstotliwość znormalizowana, gdzie $f_{n} = \frac{f}{f_{p}}$ \\
$t$                             & zmienna reprezentująca czas \\
$i$                             & zmienna reprezentująca numer próbki \\
$\gamma$                        & poziom ufności wyrażony w procentach, gdzie $\gamma = (1 - \alpha) \cdot \qty{100}{\percent}$ \\
$w(n)$                          & funkcja opisująca okno pomiarowe dla $n$-tej wielkości wejściowej \\
$s(t)$                          & zmienna w czasie fizyczna wielkość mierzona \\
$y(t)$                          & zmienna w czasie wielkość wyjściowa obiektu \\
$x(i)$                          & dyskretny przebieg wielkości wejściowej obiektu \\
$X(i)$                          & dyskretny przebieg wielkości wyjściowej obiektu \\
$\psi(t)$                       & równanie falki-matki określone w dziedzinie czasu \\
$\phi(t)$                       & równanie falki-ojca określone w dziedzinie czasu \\
$\dot{a}(x)$                    & idealny przebieg wielkości $a$ w funkcji parametru $x$ \\
$\tilde{a}(x)$                  & rzeczywisty przebieg wielkości $a$ w funkcji parametru $x$ \\
$\hat{a}(x)$                    & wartość realizacji wielkości $a$ dla wartości parametru $x$ \\
$f_{a}(x)$                      & funkcja przetwarzania statycznego obiektu $a$ \\
$G_{a}(j\omega)$                & transmitancja obiektu $a$ w dziedzinie $\mathcal{F}$ \\
$H_{a}(z)$                      & transmitancja obiektu $a$ w dziedzinie $\mathcal{Z}$ \\
$E_{a}(\omega)$                 & amplituda harmonicznej sygnału $a$ o pulsacji $\omega$ \\
$K_{a}(\omega)$                 & wzmocnienie harmonicznej sygnału $a$ o pulsacji $\omega$ \\
$\varphi_{a}(\omega)$           & przesunięcie w fazie harmonicznej sygnału $a$ o pulsacji $\omega$ \\
$N_{a}$                         & liczba wielkości wejściowych obiektu $a$ \\
$M_{a}$                         & liczba wielkości wyjściowych obiektu $a$ \\
$S_{i,j}$                       & aproksymacje $j$-tej wielkości wyjściowej dla $i$-tego poziomu dekompozycji \\
$T_{i,j}$                       & detale $j$-tej wielkości wyjściowej dla $i$-tego poziomu dekompozycji \\
$N_{q}$                         & rozdzielczość przetwornika analogowo-cyfrowego \\
$N_{k}$                         & liczba niezerowych współczynników skalujących \\
$c_{k}$                         & wartość $k$-tego współczynnika skalującego \\
$\sigma_{a}$                    & odchylenie standardowe wielkości $a$ \\
$\sigma_{a}^{2}$                & wariancja wielkości $a$ \\
$U_{a}$                         & niepewność rozszerzona związana z wielkością $a$ \\
$r_{a,b}$                       & współczynnik korelacji wielkości $a$ oraz $b$ \\
$h_{a,b}$                       & współczynnik koherencji wielkości $a$ oraz $b$ \\
$s_{a,b}$                       & współczynnik kształtu dla rozkładu realizacji sumy wielkości $a$ oraz $b$ \\
$p_{a,b}$                       & korekta współczynnika kształtu $s_{a,b}$ wynikająca z rozbieżności wartości niepewności rozszerzonej wielkości $a$ oraz $b$ \\
$k_{a,b}$                       & korekta współczynnika kształtu $s_{a,b}$ wynikająca z założeń centralnego twierdzenia granicznego \\
$a_{b,c}$                       & parametr $a$ wielkości $b$, gdzie symbol indeksu $c$ oznacza między innymi: \newline
                                  \begin{tabular}{ *{3}{l @{~--~} l} }
                                  $s$ & statyczny   & $d$      & dynamiczny & $r$      & losowy     \\
                                  $q$ & kwantowania & $z$      & zaokrągleń & $n$      & szumów     \\
                                  $p$ & propagowany & $w$      & własny     & $\Sigma$ & wypadkowy
                                  \end{tabular} \\
$c_{a}$                         & współczynnik rozszerzenia rozkładu $a$, gdzie $a$ to rozkład: \newline
                                  \begin{tabular}{ *{3}{l @{~--~} l} }
                                  $n$ & normalny    & $u$      & jednostajny & $t$      & trójkątny  \\
                                  $s$ & studenta    & $d$      & dwumodalny  & $\Sigma$ & wypadkowy
                                  \end{tabular} \\
\end{longtable}
