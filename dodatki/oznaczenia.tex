\chapter*{Wykaz oznaczeń}\addcontentsline{toc}{chapter}{Wykaz oznaczeń}

\begin{longtable}[l]{ l @{~~--~~} p{368pt} } 
$t$                             & zmienna reprezentująca czas \\ 
$\alpha$                        & założony poziom ufności \\
$\omega$                        & pulsacja przebiegu sinusoidalnego \\
$\omega_{p}$                    & pulsacja próbkowania sygnału \\
$\omega_{n}$                    & pulsacja znormalizowana do połowy pulsacji próbkowania \\
$f_{p}$                         & częstotliwość próbkowania sygnału \\
$f_{n}$                         & częstotliwość znormalizowana do połowy częstotliwości próbkowania \\
$c_{k}$                         & wartość $k$-tego współczynnika skalującego \\
$w(n)$                          & funkcja opisująca zastosowane okno pomiarowe \\ 
$s(t)$                          & zmienna w czasie fizyczna wielkość mierzona \\
$y(t)$                          & zmienna w czasie wielkość wyjściowa obiektu \\
$x(i)$                          & dyskretne przebieg wielkości wejściowej obiektu \\
$X(i)$                          & dyskretne przebieg wielkości wyjściowej obiektu \\
$\psi(t)$                       & równanie falki-matki określone w dziedzinie czasu \\
$\phi(t)$                       & równanie falki-ojca określone w dziedzinie czasu \\
$\dot{a}(t)$                    & wartość idealna wielkości $a$ w chwili $t$ \\
$\tilde{a}(t)$                  & wartość wielkości $a$ w chwili $t$ obarczona błędem \\
$N$                             & liczba wielkości wejściowych obiektu \\
$M$                             & liczba wielkości wyjściowych obiektu \\
$T_{p}$                         & okres próbkowania sygnału \\
$N_{q}$                         & rozdzielczość przetwornika analogowo-cyfrowego \\
$N_{k}$                         & liczba niezerowych współczynników skalujących \\
$A_{s,i}$                       & współczynnik przenoszenia błędów statycznych $i$-tej wielkości wyjściowej \\
$A_{r,i}$                       & współczynnik przenoszenia błędów losowych $i$-tej wielkości wyjściowej \\
$\sigma_{a}$                    & odchylenie standardowe wielkości $a$ \\
$\sigma_{a}^{2}$                & wariancja wielkości $a$ \\
$U_{a}$                         & niepewność rozszerzona wielkości $a$ \\
$r_{a,b}$                       & współczynnik korelacji wielkości $a$ oraz $b$ \\
$h_{a,b}$                       & współczynnik koherencji niepewności $a$ oraz $b$ \\
$G_{a}(j\omega)$                & transmitancja obiektu $a$ w dziedzinie $\mathcal{F}$ \\
$H_{a}(z)$                      & transmitancja obiektu $a$ w dziedzinie $\mathcal{Z}$ \\
$f_{a}(x)$                      & funkcja przetwarzania statycznego obiektu $a$ \\
$E_{a}(\omega)$                 & amplituda harmonicznej sygnału $a$ o pulsacji $\omega$ \\
$K_{a}(\omega)$                 & wzmocnienie harmonicznej sygnału $a$ o pulsacji $\omega$ \\
$\varphi_{a}(\omega)$           & przesunięcie w fazie harmonicznej sygnału $a$ o pulsacji $\omega$ \\
$S_{i,j}$                       & aproksymacje $j$-tej wielkości wyjściowej $i$-tego poziomu dekompozycji \\
$T_{i,j}$                       & detale $j$-tej wielkości wyjściowej $i$-tego poziomu dekompozycji \\
$a_{b,c}$                       & parametr $a$ wielkości $b$, gdzie symbol indeksu $c$ oznacza: \newline
                                  \begin{tabular}{ *{3}{l @{~--~} l} }
                                  $s$ & statyczny   & $d$      & dynamiczny & $r$      & losowy     \\
                                  $p$ & propagowany & $w$      & własny     & $\Sigma$ & wypadkowy  \\
                                  $q$ & kwantowania & $z$      & zaokrągleń & $n$      & szumów
                                  \end{tabular} \\
$c_{a}$                         & współczynnik rozszerzenia rozkładu $a$, gdzie $a$ to rozkład: \newline
                                  \begin{tabular}{ *{3}{l @{~--~} l} }
                                  $n$ & normalny    & $u$      & jednostajny & $t$      & trójkątny  \\
                                  $s$ & studenta    & $d$      & dwumodalny  & $\Sigma$ & wypadkowy  
                                  \end{tabular} \\
\end{longtable}

