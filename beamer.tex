\documentclass[12pt, polish]{beamer}
\usepackage{beamer}

\date{TODO}
\title{Rozprawa Doktorska}
\author{mgr inż. Łukasz Dróżdż}
\subject{Analiza metrologiczna algorytmów dyskretnej transformacji falkowej}
\subtitle{Analiza metrologiczna algorytmów dyskretnej transformacji falkowej}
\institute{Politechnika Śląska, Wydział Elektryczny \\ Katedra Metrologii, Elektroniki i Automatyki}
\keywords{dyskretna transformacja falkowa, cyfrowe przetwarzanie sygnałów, szacowanie niepewności wyniku pomiaru, analiza właściwości metrologicznych toru pomiarowego}

\begin{document}

\section*{Wstęp}

\begin{frame}
\titlepage
\end{frame}

\begin{frame}{Plan prezentacji}
\tableofcontents
\end{frame}

\section{Przedstawienie tezy pracy oraz jej najważniejszych założeń}

\begin{frame}{Teza pracy}
\justifying
Stosując przedstawiony w pracy model błędów oraz zaproponowaną metodę szacowania wypadkowej wartości niepewności rozszerzonej istnieje możliwość oszacowania wartości niepewności rozszerzonych dla wielkości wyjściowych toru pomiarowego wykorzystującego algorytm dyskretnej transformacji falkowej. Oszacowanie wartości niepewności rozszerzonych dla omawianych wielkości jest możliwe w trakcie działania systemu pomiarowego, również w przypadku zmiany parametrów pracy tego systemu i wynikającej z tego tytułu zmiany parametrów modelu błędów. Skuteczność zaproponowanej metody szacowania wypadkowej wartości niepewności rozszerzonej wynika z dokładności określenia parametrów modelu błędów, a uzyskiwane wyniki są zbieżne z uzyskiwanymi metodą Monte-Carlo.
\end{frame}

\begin{frame}{Podsumowanie tezy}
\begin{itemize}
\item Propozycja modelu błędów umożliwiającego opis właściwości metrologicznych torów pomiarowych wykorzystujących algorytmy transformacji falkowej
\item Propozycja metody szacowania wartości wypadkowej niepewności rozszerzonej, która:
	\begin{itemize}
	\item cechuje się niską złożonością obliczeniową i jest możliwa do stosowania w czasie pracy systemu pomiarowego
	\item daje możliwość zmiany parametrów modelu błędów podczas pracy systemu pomiarowego
	\end{itemize}
\item Wyniki uzyskiwane przy użyciu zaproponowanej metody powinny być zbieżne z uzyskiwanymi metodą Monte-Carlo
\item Dokładność uzyskiwanych wyników może zależeć od dokładności wyznaczenia parametrów modelu błędu
\end{itemize}
\end{frame}

\section{Przedstawienie najważniejszych właściwości metrologicznych algorytmów transformacji falkowej}

\begin{frame}{TODO}
TODO
\end{frame}

\section{Omówienie najważniejszych założeń odnośnie zaproponowanego w pracy modelu błędów}

\begin{frame}{TODO}
TODO
\end{frame}

\section{Omówienie zaproponowanej metody szacowania wypadkowej wartości niepewności rozszerzonej}

\begin{frame}{TODO}
TODO
\end{frame}

\section{Przedstawienie wyników badań symulacyjnych i pomiarowych, weryfikujących zasadność tezy pracy}

\begin{frame}{TODO}
TODO
\end{frame}

\section{Przedstawienie najważniejszych wniosków płynących z pracy}

\begin{frame}{TODO}
TODO
\end{frame}

\section*{Zakończenie}

\begin{frame}{Dziękuję za uwagę}
\centering
mgr inż. Łukasz Dróżdż \\ \href{mailto:lukasz.drozdz@polsl.pl}{lukasz.drozdz@polsl.pl}
\vskip 16pt
Politechnika Śląska, Wydział Elektryczny \\ Katedra Metrologii, Elektroniki i Automatyki
\end{frame}

\end{document}
