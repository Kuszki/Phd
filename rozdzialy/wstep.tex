\chapter{Wstęp do pracy}

Ze względu na dynamiczny rozwój mikrokontrolerów, zwiększenie stopnia integracji oraz możliwości obliczeniowych tych układów, coraz częściej w systemach pomiarowo-sterujących stosowane są rozwiązania \enquote{SoC} (ang. \enquote{System on Chip})~\cite{saleh_systemonchip}. Opisywane podejście jest korzystne ze względu na niższy koszt projektowanego systemu, a także ze względu na łatwiejszy proces jego projektowania, zmniejszenie potrzebnej przestrzeni (miniaturyzacje układu), jak również eliminacje konieczności uwzględniania w projekcie wielu dodatkowych urządzeń. Dostępne na rynku mikrokontrolery integrują w sobie niemal wszystkie potrzebne układy peryferyjne: pamięć \enquote{RAM}, pamięć \enquote{FLASH} oraz interfejsy komunikacyjne (\enquote{USART}, \enquote{I2C}, \enquote{I2S}, \enquote{SPI}, \enquote{CAN}, \enquote{USB}, \enquote{SDIO}, \enquote{Ethernet}). Dodatkowo omawiane układy są często wyposażone w zestaw instrukcji oferujący funkcje \enquote{DSP} (ang. \enquote{Digital Signal Processor}), które w połączeniu z układem \enquote{FPU} zapewniają bardzo dobrą wydajność podczas obliczeń zmiennoprzecinkowych. W powyższych okolicznościach mikrokontrolery te spełniają oczekiwania stawiane systemom pomiarowo-sterującym, w których implementuje się różnego rodzaju algorytmy przetwarzania danych. Schemat blokowy toru pomiarowego wykorzystującego omawiane rozwiązanie przedstawiono na rysunku~\ref{fig:chain_demo}, gdzie $s(t)$ oznacza fizyczną wielkość mierzoną w czasie, $y(t)$ wielkość wyjściową przetwornika pomiarowego, $x(i)$ wielkość wejściową algorytmu przetwarzania danych oraz $\mathbf{X}(i)$ kolejne wektory wielkości wyjściowych toru pomiarowego.

\begin{figure}[htb!]
\begin{center}
\includegraphics{obrazki/schemat_demo}
\makecaption{fig:chain_demo}{Schemat blokowy przykładowego toru pomiarowego, w którym wykorzystano rozwiązanie typu \enquote{System on Chip}}
\end{center}
\end{figure}

Pierwsze informacje na temat falek pojawiły się około 1909 roku i zostały opublikowane przez Alfréda Haara, węgierskiego matematyka~\cite{haar_basics}. Falka Hara posiadała jednak kilka istotnych wad: ze względu na swoją nieciągłość była nieróżniczkowalna, co dyskwalifikowało jej użycie w pewnych okolicznościach. Podobne badania prowadził połowie XX wieku brytyjsko-węgierski naukowiec Dennis Gabor. Analiza falkowa została po raz pierwszy opisana przez francuskiego geofizyka Jeana Morleta, przy czym algorytm transformacji falkowej został opisany w 1988 roku przez francuskiego naukowca Stéphane Mallata. We wczesnych latach 90. XX wieku transformacja falkowa oraz rozwój falek były zagadnieniem bardzo popularnym. Tematy związane z transformacją falkową są dyskutowane i rozwijane do dzisiaj, a obszar zastosowań tych algorytmów stale się powiększa~\cite{akujuobi_applications}.

Ze względu na fakt, że algorytmy transformacji falkowej stanowią istotną część toru pomiarowego, ich analiza nie może zostać pominięta w ocenie właściwości metrologicznych tego toru. Każdy element toru pomiarowego musi być poddany analizie metrologicznej, aby można było ilościowo określić niedokładność wyznaczania wartości wielkości wyjściowych tego toru. Ze względu na mnogość dostępnych falek, należy przedstawić jednolitą i uniwersalną metodę pozwalającą na oszacowanie wartości niepewności wielkości wyjściowych tych algorytmów przy założeniu, że znane są parametry sygnałów błędów wielkości wejściowych algorytmu. Duży stopień integracji rozwiązań \enquote{SoC} powoduje jednak, że w większości przypadków dokładny model toru pomiarowego nie jest znany. Powyższe okoliczności sprawiają, że analiza metrologiczna torów pomiarowych wykorzystujących algorytmy transformacji falkowej jest często pomijana. Proponuje się zatem następującą tezę:

\begin{quoting}[font = bfseries]
Stosując przedstawiony w pracy model błędów oraz zaproponowaną metodę szacowania wypadkowej wartości niepewności rozszerzonej istnieje możliwość oszacowania wartości niepewności rozszerzonych dla wielkości wyjściowych toru pomiarowego wykorzystującego algorytm dyskretnej transformacji falkowej. Oszacowanie wartości niepewności rozszerzonych dla omawianych wielkości jest możliwe w trakcie działania systemu pomiarowego, również w przypadku zmiany parametrów pracy tego systemu i wynikającej z tego tytułu zmiany parametrów modelu błędów. Skuteczność zaproponowanej metody szacowania wypadkowej wartości niepewności rozszerzonej wynika z dokładności określenia parametrów modelu błędów, a uzyskiwane wyniki są zbieżne z uzyskiwanymi metodą Monte-Carlo.
\end{quoting}

Przedstawiona powyżej teza zakłada, że znajomość dokładnego modelu błędów analizowanego toru pomiarowego nie jest konieczna, natomiast w przypadku pominięcia istotnych właściwości tego toru podczas określania parametrów zaproponowanego w pracy modelu błędów, uzyskane wyniki mogą być niedokładne. Parametry zaproponowanego w pracy modelu błędów pozyskać można na drodze identyfikacji właściwości toru pomiarowego w sposób eksperymentalny, lub jeśli to możliwe, pozyskać je na podstawie dokumentacji kolejnych elementów toru pomiarowego. Dokładność identyfikacji parametrów modelu błędów ma kluczowy wpływ na dokładność oszacowania wartości niepewności rozszerzonych wielkości wyjściowych toru pomiarowego, wobec czego rolą projektanta toru pomiarowego będzie przyjęcie odpowiednich założeń odnośnie wprowadzonych uproszczeń.

Wymagania stawiane zaproponowanej w pracy metodzie szacowania wypadkowej wartości niepewności rozszerzonej powodują, że metoda ta musi cechować się niską złożonością obliczeniową oraz musi dopuszczać możliwość zmiany parametrów zastosowanego modelu błędów. Wymaganie to zostało wprowadzone z uwagi na fakt, że wypadkowa wartość niepewności rozszerzonej zależeć może nie tylko od właściwości analizowanego toru pomiarowego, ale i od parametrów przetwarzanego sygnału oraz zmiennych warunków otoczenia. Proponowana metoda powinna zapewniać wyniki zbliżone do tych uzyskiwanych symulacyjnie metodą Monte-Carlo, a jednocześnie być możliwa do realizacji w czasie rzeczywistym w analizowanym torze pomiarowym.

Praca została podzielona na \total{chapter} rozdziałów. Rozdział pierwszy stanowi wstęp do pracy i zawiera jej tezę. W rozdziale drugim opisany został zaproponowany model błędów dla kolejnych fragmentów toru pomiarowego oraz przedstawione zostały związki pomiędzy definiowanymi sygnałami błędów. Rozdział trzeci stanowi charakterystykę algorytmu transformacji falkowej i zawiera najważniejsze informacje dotyczące właściwości tego algorytmu. W rozdziale czwartym przedstawione zostały wyniki badań symulacyjnych, weryfikujące skuteczność zaproponowanej metody szacowania niepewności wielkości wyjściowych algorytmów dyskretnej transformacji falkowej. Rozdział piąty stanowi weryfikacje pomiarową przedstawionej tezy wraz z przykładem identyfikacji właściwości analizowanego toru pomiarowego, przy czym wartości uzyskane na drodze eksperymentu pomiarowego są porównane z wynikami otrzymanymi przy zastosowaniu zaproponowanej w pracy metody. W ostatnim rozdziale zawarto podsumowanie pracy i sformułowano najważniejsze wnioski płynące z jej treści.

Motyw przewodni pracy stanowią właściwości metrologiczne analizowanych algorytmów, ich rola podczas przetwarzania i wprowadzania zdefiniowanych sygnałów błędów, opis zaproponowanego modelu błędów oraz weryfikacja założeń wynikających z przedstawionej tezy. Zawarte w pracy informacje na temat algorytmów transformacji falkowej, konieczne do analizy ich właściwości metrologicznych, stanowią podsumowanie i zestawienie najważniejszych zależności zawartych w literaturze i nie stanowią dorobku autora pracy. W ujęciu pracy algorytm transformacji falkowej jest zatem narzędziem, którego analiza właściwości metrologicznych jest przeprowadzana w celu oceny stopnia realizacji określonego zadania pomiarowego przez tor pomiarowy wykorzystujący to narzędzie. Algorytmy transformacji falkowej mogą przetwarzać jedno lub dwuwymiarowe dane wejściowe, natomiast praca skupia się na algorytmach przetwarzających jednowymiarowe ciągi danych wejściowych~\cite{wallen_handbook}. Ze względu na fakt, że dyskretna odmiana algorytmu transformacji falkowej jest stosowana w torach pomiarowo-sterujących częściej, praca poświęca najwięcej uwagi tej wersji algorytmu. Praca ogranicza się do rodzin falek o rzeczywistych wartościach współczynników oraz torów pomiarowych o wielkościach wejściowych z dziedziny liczb rzeczywistych.

Jako główną miarę dla określania niedokładności z jaką wyznaczana jest wartość analizowanej wielkości wyjściowej przyjęto w pracy niepewność rozszerzoną~\cite{jcgm_guide}, przy czym równolegle przedstawiany jest opis wykorzystujący miarę wariancji sygnału błędu. Zawarte w pracy przykłady ograniczają się do przypadków mieszczących się w zakresie klasycznej definicji niepewności rozszerzonej~\cite{jcgm_guide}.
