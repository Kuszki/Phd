\chapter{Wstęp do pracy}

Ze względu na dynamiczny rozwój mikrokontrolerów, zwiększenie stopnia integracji oraz możliwości obliczeniowych tych układów, coraz częściej w systemach pomiarowo-sterujących stosowane są rozwiązania typu \enquote{SoC} (ang. \enquote{System on Chip})~\cite{saleh_systemonchip}. Opisywane podejście jest korzystne ze względu na niższy koszt projektowanego systemu, a także ze względu na łatwiejszy proces jego projektowania, zmniejszenie potrzebnej przestrzeni (miniaturyzacje układu), jak również eliminacje konieczności uwzględniania w projekcie wielu dodatkowych urządzeń. Dostępne na rynku mikrokontrolery integrują w sobie niemal wszystkie potrzebne układy peryferyjne: pamięć \enquote{RAM}, pamięć \enquote{FLASH} oraz interfejsy komunikacyjne (\enquote{USART}, \enquote{I2C}, \enquote{I2S}, \enquote{SPI}, \enquote{CAN}, \enquote{USB}, \enquote{SDIO}, \enquote{Ethernet}). Dodatkowo omawiane układy są często wyposażone w zestaw instrukcji oferujący funkcje \enquote{DSP}, które w połączeniu ze zintegrowanym układem \enquote{FPU} zapewniają bardzo dobrą wydajność podczas obliczeń zmiennoprzecinkowych~\cite{reay_dsp}. W powyższych okolicznościach mikrokontrolery te spełniają oczekiwania stawiane systemom pomiarowo-sterującym, w których implementuje się różnego rodzaju algorytmy przetwarzania danych~\cite{saleh_systemonchip}. Schemat blokowy toru pomiarowego wykorzystującego omawiane rozwiązanie przedstawiono na rysunku~\ref{fig:chain_demo}, przy czym symbolem $s(t)$ oznaczono zmienną w czasie fizyczną wielkość mierzoną, $y(t)$ wielkość wyjściową przetwornika pomiarowego, $x(i)$ wielkość wejściową algorytmu przetwarzania danych, natomiast $\mathbf{X}(i)$ kolejne wektory wielkości wyjściowych toru pomiarowego wyznaczone na podstawie pozyskanych próbek wielkości $x(i)$.

\begin{figure}[htb!]
\begin{center}
\includegraphics{obrazki/schemat_demo}
\makecaption{fig:chain_demo}{Schemat blokowy przykładowego toru pomiarowego, w którym wykorzystano rozwiązanie typu \enquote{System on Chip}}
\end{center}
\end{figure}

Pierwsze informacje na temat falek pojawiły się około 1909 roku i zostały opublikowane przez Alfréda Haara, węgierskiego matematyka~\cite{haar_basics}. Falka Hara posiadała jednak kilka istotnych wad: ze względu na swoją nieciągłość była nieróżniczkowalna, co dyskwalifikowało jej użycie w pewnych okolicznościach. Podobne badania prowadził połowie XX wieku brytyjsko-węgierski naukowiec Dennis Gabor. Analiza falkowa została po raz pierwszy opisana przez francuskiego geofizyka Jeana Morleta, przy czym algorytm transformacji falkowej został opisany w 1988 roku przez francuskiego naukowca Stéphane Mallata. We wczesnych latach 90. XX wieku transformacja falkowa oraz rozwój falek były zagadnieniem bardzo popularnym. Tematy związane z transformacją falkową są dyskutowane i rozwijane do dzisiaj, a obszar zastosowań tych algorytmów stale się powiększa~\cite{akujuobi_applications}.

Ze względu na fakt, że algorytmy transformacji falkowej stanowią istotną część toru pomiarowego, ich analiza nie może zostać pominięta w ocenie właściwości metrologicznych tego toru. Każdy element toru pomiarowego musi być poddany analizie metrologicznej, aby można było ilościowo określić niedokładność wyznaczania wartości wielkości wyjściowych tego toru. Ze względu na mnogość dostępnych falek, należy przedstawić jednolitą i uniwersalną metodę pozwalającą na oszacowanie wartości niepewności wielkości wyjściowych tych algorytmów przy założeniu, że znane są parametry sygnałów błędów wielkości wejściowych algorytmu. Duży stopień integracji rozwiązań \enquote{SoC} powoduje jednak, że w większości przypadków dokładny model toru pomiarowego nie jest znany. Wskazane okoliczności sprawiają, że analiza metrologiczna torów pomiarowych wykorzystujących algorytmy transformacji falkowej jest często pomijana, jak to miało miejsce w pracach~\cite{wallen_handbook, anping_seismic, yan_mechanics, niedopytalski_ene, niedopytalski_zwar, xie_metals}.

Najbardziej uniwersalną metodą, która może być stosowana w celu oszacowania miary niedokładności wyznaczania wartości wielkości wyjściowych toru pomiarowego, jest metoda Monte-Carlo~\cite{jcgm_montecarlo, janssen_montecarlo}. Metoda ta wymaga jednak wykonania czasochłonnego eksperymentu, co stanowi problem w przypadku zmiany parametrów modelu błędów. Należy zauważyć, że parametry eksperymentu zmieniają się również w przypadku zmiany widma przetwarzanego przez tor pomiarowy sygnału wejściowego, czy warunków otoczenia. Dodatkowo do przeprowadzenia eksperymentu konieczna jest implementacja stosowanego algorytmu transformacji falkowej o wybranych parametrach.

Powszechnie używaną miarą niedokładności wyniku pomiaru jest niepewność standardowa~\cite{jcgm_guide}. W kontekście analizy parametrów sygnału błędu posługiwać się można również miarą wariancji tego sygnału, gdzie w przypadku sygnałów błędów o zerowej wartości oczekiwanej, wariancję tych sygnałów utożsamiać można z ich mocą~\cite{oppenheim_sns}. W przypadku miary niepewności standardowej, której wartość wynika z parametrów analizowanego sygnału błędu, nie zawsze istnieje możliwość wskazania prawdopodobieństwa wystąpienia wybranej wartości realizacji tego sygnału, stąd znacznie częściej stosowana jest miara niepewności rozszerzonej. Zakłada się, że proces pomiaru jest powtarzany wielokrotnie podczas pracy systemu pomiarowego.

Przedstawione rozważania podkreślają potrzebę sporządzenia jednolitego modelu błędów, który uwzględniając właściwości toru pomiarowego opisze relacje pomiędzy istniejącymi w nim sygnałami błędów oraz umożliwi ilościowy opis parametrów wypadkowego sygnału błędu na wyjściu tego toru. Zaproponowany model błędów musi być odpowiedni dla torów pomiarowych, w których występują algorytmy transformacji falkowej. Ponadto istnieje potrzeba przedstawienia metody szacowania wypadkowej wartości niepewności rozszerzonej, która cechować się będzie niską złożonością obliczeniową, zapewniającą możliwość stosowania tej metody w czasie rzeczywistym, w przypadku zmiany parametrów modelu błędów. Aplikacja modelu błędów oraz zaproponowanej metody analizy powinna być możliwa również w przypadku, gdy projektant toru pomiarowego nie jest ekspertem z dziedziny algorytmów transformacji falkowej. Uwzględniając wymienione dotychczas argumenty, stanowiące motywacje pracy, proponuje się następującą tezę:
\begin{quoting}[font = bfseries]
Stosując przedstawiony w pracy model błędów oraz zaproponowaną metodę szacowania wypadkowej wartości niepewności rozszerzonej istnieje możliwość oszacowania wartości niepewności rozszerzonych dla wielkości wyjściowych toru pomiarowego wykorzystującego algorytm dyskretnej transformacji falkowej. Oszacowanie wartości niepewności rozszerzonych dla omawianych wielkości jest możliwe w trakcie działania systemu pomiarowego, również w przypadku zmiany parametrów pracy tego systemu oraz zmiany parametrów modelu błędów.
\end{quoting}

Wymagania stawiane zaproponowanej w pracy metodzie szacowania wypadkowej wartości niepewności rozszerzonej powodują, że metoda ta musi cechować się niską złożonością obliczeniową oraz musi dopuszczać możliwość zmiany parametrów zastosowanego modelu błędów. Wymaganie to zostało wprowadzone z uwagi na fakt, że wypadkowa wartość niepewności rozszerzonej zależeć może nie tylko od właściwości analizowanego toru pomiarowego, ale i od parametrów przetwarzanego sygnału oraz zmiennych warunków otoczenia. Proponowana metoda powinna zatem zapewniać wyniki zbliżone do tych uzyskiwanych metodą Monte-Carlo, a jednocześnie być możliwa do realizacji w czasie rzeczywistym w analizowanym systemie pomiarowym. W dalszej części pracy przyjęto, że dopuszczalna rozbieżność oszacowania wypadkowej wartości niepewności rozszerzonej nie powinna przekraczać~\qty{\pm 5}{\percent} w odniesieniu do wartości uzyskanej metodą Monte-Carlo.

Praca została podzielona na \total{chapter} rozdziałów. Rozdział pierwszy stanowi wstęp do pracy, zawiera jej tezę oraz najważniejsze założenia. W rozdziale drugim opisany został zaproponowany model błędów dla kolejnych fragmentów toru pomiarowego oraz przedstawione zostały związki zachodzące pomiędzy zdefiniowanymi sygnałami błędów. Rozdział trzeci stanowi przegląd literatury, przy czym zestawiono w nim najważniejsze informacje dotyczące algorytmów transformacji falkowej, istotne z punktu widzenia właściwości metrologicznych tych algorytmów oraz stosowania zaproponowanego modelu błędów. W rozdziale czwartym przedstawione zostały wyniki badań symulacyjnych, weryfikujące skuteczność zaproponowanej metody szacowania wartości niepewności rozszerzonych wielkości wyjściowych algorytmów dyskretnej transformacji falkowej. Rozdział piąty stanowi weryfikacje pomiarową przedstawionej tezy wraz z przykładem identyfikacji właściwości analizowanego toru pomiarowego, przy czym wyniki uzyskane na drodze eksperymentu pomiarowego porównano z wynikami otrzymanymi przy zastosowaniu zaproponowanej w pracy metody analizy. W ostatnim rozdziale zawarto podsumowanie pracy i sformułowano najważniejsze wnioski płynące z jej treści. Zakres pracy obejmuje zatem:
\begin{itemize}
\item definicję modelu błędów opisującego właściwości metrologiczne toru pomiarowego,
\item opis metody wyznaczania wypadkowej wartości niepewności rozszerzonej,
\item wskazanie właściwości metrologicznych algorytmów transformacji falkowej,
\item przedstawienie przykładu aplikacji opisanej w pracy metody analizy,
\item symulacyjną i pomiarową weryfikację skuteczności przedstawionej metody.
\end{itemize}

Motyw przewodni pracy stanowią właściwości metrologiczne analizowanych algorytmów transformacji falkowej, ich rola podczas przetwarzania i wprowadzania sygnałów błędów, opis zaproponowanego modelu błędów oraz weryfikacja założeń wynikających z przedstawionej tezy. Zawarte w pracy informacje na temat algorytmów transformacji falkowej, konieczne do analizy ich właściwości metrologicznych, stanowią podsumowanie i zestawienie najważniejszych zależności zawartych w literaturze i nie stanowią dorobku autora pracy. W ujęciu pracy algorytm transformacji falkowej jest zatem narzędziem, którego analiza właściwości metrologicznych jest przeprowadzana w celu oceny stopnia realizacji określonego zadania pomiarowego przez tor pomiarowy wykorzystujący to narzędzie. Wobec powyższego, w pracy nie rozważano właściwości tych algorytmów dla konkretnych ich aplikacji, a jedynie przedstawiono, w jaki sposób analizować metrologiczne właściwości tych algorytmów, istotne z punktu widzenia liściowej oceny niedokładności wyznaczania wartości realizacji wielkości wyjściowych torów pomiarowych stosujących te algorytmy. W pracy przyjęto założenie, że projektant toru pomiarowego nie jest ekspertem z zakresu algorytmów transformacji falkowej, a jedynie stosuje te algorytmy do realizacji wybranego zadania pomiarowego.

Algorytmy transformacji falkowej mogą przetwarzać jedno lub dwuwymiarowe dane wejściowe, natomiast praca skupia się na algorytmach przetwarzających jednowymiarowe ciągi danych wejściowych. Jako, że dyskretna odmiana algorytmu transformacji falkowej jest stosowana w torach pomiarowo-sterujących częściej, niż pozostałe wersje tego algorytmu~\cite{wallen_handbook, lord_guide, akujuobi_applications}, praca poświęca najwięcej uwagi tej wersji algorytmu, natomiast przedstawiona metodologia analizy może być stosowana również w pozostałych przypadkach. Praca ogranicza się do rodzin falek o rzeczywistych wartościach współczynników oraz torów pomiarowych o wielkościach wejściowych z dziedziny liczb rzeczywistych.

Jako główną miarę dla określania niedokładności, z jaką wyznaczane są wartości realizacji analizowanej wielkości, przyjęto w pracy niepewność rozszerzoną~\cite{jcgm_guide}, przy czym równolegle przedstawiany jest opis wykorzystujący miarę wariancji sygnału błędu związanego z tą wielkością. Zawarte w pracy przykłady ograniczają się do przypadków mieszczących się w zakresie klasycznej definicji niepewności rozszerzonej, a zatem dotyczą sygnałów błędów o zerowej wartości oczekiwanej, których funkcja gęstości prawdopodobieństwa uzyskania zadanej wartości realizacji jest symetryczna względem osi rzędnych~\cite{jcgm_guide}. W przypadku analizy sygnałów błędów o niezerowej wartości oczekiwanej proponuje się zastosowanie korekty wyniku pomiaru o wartość oczekiwaną realizacji analizowanego sygnału błędu. W przypadku sygnałów błędów o niesymetrycznym kształcie funkcji gęstości prawdopodobieństwa proponuje się osobną analizę przedziału dla ujemnych i dodatnich wartości realizacji tego sygnału, co zostało opisane między innymi w pracach~\cite{roj_annuncertainty, wymyslo_range, jakubiec_system}.

Parametry zaproponowanego w pracy modelu błędów pozyskać można na drodze identyfikacji właściwości toru pomiarowego w sposób eksperymentalny, lub jeśli to możliwe, pozyskać je na podstawie dokumentacji kolejnych elementów tego toru. Dokładność identyfikacji parametrów modelu błędów ma kluczowy wpływ na dokładność oszacowania wartości niepewności rozszerzonych wielkości wyjściowych toru pomiarowego, wobec czego rolą projektanta toru pomiarowego będzie przyjęcie odpowiednich założeń odnośnie wprowadzonych uproszczeń. W przypadku pominięcia istotnych właściwości tego toru podczas określania parametrów modelu błędów uzyskane wyniki mogą być niedokładne.
