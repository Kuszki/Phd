\chapter{Wstęp do pracy}

Ze względu na dynamiczny rozwój mikrokontrolerów, zwiększenie stopnia integracji oraz możliwości obliczeniowych tych układów, co raz częściej w systemach pomiarowo-sterujących stosowane są rozwiązania \enquote{SoC} (ang. \enquote{System on Chip}) \cite{saleh_systemonchip}. Opisywane podejście jest korzystne ze względu na koszt projektowanego systemu, a także ze względu na łatwiejszy proces jego projektowania, zmniejszenie potrzebnej przestrzeni (miniaturyzacje układu), jak również eliminacje konieczności uwzględniania w projekcie wielu dodatkowych urządzeń.

Dostępne na rynku mikrokontrolery integrują w sobie niemal wszystkie potrzebne układy peryferyjne: pamięć \enquote{RAM}, pamięć \enquote{FLASH}, interfejsy komunikacyjne (\enquote{USART}, \enquote{I2C}, \enquote{I2S}, \enquote{SPI}, \enquote{CAN}, \enquote{USB}, \enquote{SDIO}, \enquote{Ethernet}). Dodatkowo omawiane układy są często wyposażone w zestaw instrukcji oferujący funkcje \enquote{DSP}, które w połączeniu z układem \enquote{FPU} zapewniają bardzo dobrą wydajność podczas obliczeń zmiennoprzecinkowych. W powyższych okolicznościach mikrokontrolery te spełniają oczekiwania stawiane systemom pomiarowo-sterującym, w których implementuje się różnego rodzaju algorytmy przetwarzania danych. Schemat ideowy toru pomiarowego wykorzystującego omawiane rozwiązanie przedstawiono na rysunku \ref{fig_chain_demo}, gdzie $s(t)$ oznacza fizyczną wielkość mierzoną w czasie, $y(t)$ wartość sygnału przetwornika pomiarowego, $x(i)$ kolejne próbki sygnału, a $X(j)$ kolejne wielkości wyjściowe toru pomiarowego.

\begin{figure}[htb!]
\begin{center}
\includegraphics{obrazki/schemat_demo}
\caption{Schemat blokowy przykładowego toru pomiarowego wykorzystującego omawiane rozwiązanie \enquote{System on Chip} \label{fig_chain_demo}}
\end{center}
\end{figure}

Pierwsze informacje na temat falek pojawiły się około 1909 roku opublikowane przez Alfréda Haara, węgierskiego matematyka \cite{haar_basics}. Falka Hara posiadała jednak kilka istotnych wad: ze względu na swoją nieciągłość była nieróżniczkowalna, co dyskwalifikowało jej użycie w pewnych okolicznościach. Podobne badania prowadził połowie XX wieku brytyjsko-węgierski naukowiec Dennis Gabor. Analiza falkowa została po raz pierwszy opisana przez francuskiego geofizyka Jeana Morleta, przy czym algorytm transformacji falkowej został opisany w 1988 roku przez francuskiego naukowca Stéphane Mallata. We wczesnych latach 90. XX wieku transformacja falkowa oraz rozwój falek były zagadnieniem bardzo popularnym. Tematy związane z transformacją falkową są dyskutowane i rozwijane do dzisiaj, a obszar zastosowań tych algorytmów stale się powiększa \cite{akujuobi_applications}.

Ze względu na fakt, że algorytmy transformacji falkowej stanowią istotną część toru pomiarowego, ich analiza nie może zostać pominięta w ocenie właściwości metrologicznych tego toru. Każdy element toru pomiarowego musi być poddany analizie metrologicznej, aby na podstawie właściwości wszystkich jego elementów można było oszacować niepewność wielkości wyjściowych tego toru pomiarowego. Ze względu na mnogość dostępnych falek i stopień ich skomplikowania, należy przedstawić jednolitą, uniwersalną metodę pozwalającą na oszacowanie niepewności wielkości wyjściowych tych algorytmów przy założeniu, że znane są parametry rozkładu błędów ich wielkości wejściowych. Duży stopień integracji rozwiązań \enquote{SoC} powoduje jednak, że w większości przypadków dokładny model fragmentu toru pomiarowego stanowionego przez te układy nie jest znany. Na podstawie powyższych informacji sformułować można tezę pracy:

\begin{quoting}[font = bfseries]
Stosując przedstawiony w pracy model błędu dla kolejnych fragmentów toru pomiarowego, a następnie identyfikując pomiarowo lub pozyskując na podstawie dokumentacji układu jego parametry, istnieje możliwość oszacowania niepewności wielkości wyjściowych toru pomiarowego wykorzystującego algorytm dyskretnej transformacji falkowej. Opisywaną analizę można przeprowadzić traktując tor pomiarowy jako jeden element lub przeprowadzając osobno analizę każdego elementu, a następnie łącząc finalnie wszystkie elementy ze sobą.
\end{quoting}

Przedstawiona teza zakłada, że znajomość dokładnego modelu toru pomiarowego nie jest konieczna -- niezbędne informacje można pozyskać na drodze identyfikacji jego właściwości w sposób eksperymentalny, lub jeśli to możliwe -- pozyskać je z dokumentacji producenta układu. Jakość opracowanego modelu i dokładność identyfikacji jego parametrów będzie miała kluczowy wpływ na dokładność szacowania niepewności wielkości wyjściowych, wobec czego rolą projektanta toru pomiarowego będzie przyjęcie odpowiednich założeń odnośnie wprowadzonych uproszczeń.

Algorytmy transformacji falkowej mogą przetwarzać jedno lub dwuwymiarowe dane wejściowe, natomiast praca skupiać się będzie na algorytmach przetwarzających jednowymiarowe szeregi danych wejściowych. Ze względu na fakt, że dyskretna odmiana algorytmu transformacji falkowej jest stosowana w torach pomiarowo-sterujących częściej, praca poświęci najwięcej uwagi tej wersji algorytmu. Praca ograniczać się będzie do rodzin falek o rzeczywistych wartościach współczynników oraz torów pomiarowych o wielościach wejściowych z dziedziny liczb rzeczywistych. Głównym celem pracy jest przedstawienie uniwersalnej metody szacowania niepewności wielkości wyjściowych algorytmów dyskretnej transformacji falkowej.

Praca została podzielona na \total{chapter} rozdziałów. Rozdział pierwszy stanowi wstęp do pracy i zawiera jej najważniejszą tezę. W rozdziale drugim opisany został zaproponowany model błędu fragmentu toru pomiarowego oraz przedstawiony został algorytm identyfikacji jego parametrów. Rozdział trzeci stanowi charakterystykę algorytmu transformacji falkowej i zawiera najważniejsze informacje dotyczące właściwości tego algorytmu. W rozdziale czwartym przedstawione zostaną wyniki badań symulacyjnych, weryfikujące skuteczność zaproponowanej metody szacowania niepewności wielkości wyjściowych algorytmów dyskretnej transformacji falkowej. Rozdział piąty stanowić będzie weryfikacje pomiarową przedstawionej tezy, gdzie wartości uzyskane na drodze eksperymentu pomiarowego zostaną porównane z wynikami otrzymanymi przy zastosowaniu zaproponowanej w pracy metody. W ostatnim rozdziale zawarto podsumowanie pracy i sformułowano najważniejsze wnioski płynące z jej treści.

Motywem przewodnim pracy będą właściwości metrologiczne analizowanych algorytmów, ich rola w torach pomiarowych oraz opis zaproponowanego modelu błędu, a następnie weryfikacja prawdziwości przedstawionej tezy. Zawarte w pracy informacje na temat algorytmów transformacji falkowej, konieczne do analizy ich właściwości metrologicznych, będą podsumowaniem i zestawieniem najważniejszych zależności zawartych w literaturze i nie stanowią dorobku autora pracy. W ujęciu pracy algorytm transformacji falkowej będzie zatem narzędziem, którego analiza właściwości metrologicznych zostanie przeprowadzona w celu oceny realizacji określonego zadania pomiarowego przez tor pomiarowy wykorzystujący to narzędzie.

