\chapter{Pomiarowa weryfikacja tezy}

Ostatnim etapem weryfikacji postawionej w pracy tezy będzie analiza właściwości metrologicznych przykładowego toru pomiarowego, który stworzony został na potrzeby pracy. Analizowany tor pomiarowy przetwarzać będzie zmienny w czasie sygnał napięciowy $s(t)$ o wartości napięcia z przedziału $<0;1>~\unit{V}$. W celu dopasowania poziomu sygnału wejściowego do zakresu napięcia wejściowego przetwornika analogowo-cyfrowego, zwiększenia impedancji wejściowej toru pomiarowego oraz zmniejszeniu impedancji obwodu analogowo-cyfrowego, sygnał $s(t)$ podawany będzie na wejście wzmacniacza operacyjnego. Sygnał wyjściowy wzmacniacza, oznaczony $y(t)$ będzie przetwarzany na postać cyfrową, oznaczoną $x(i)$, a następnie podawany będzie na wejście jednostki \enquote{DSP}, która pobierać będzie $N$ próbek tego sygnału i wyznaczać na ich podstawie $M$ wartości wielkości wyjściowych toru pomiarowego, stosując w tym celu algorytm dyskretnej transformacji falkowej. Schemat blokowy omawianego toru pomiarowego przedstawiono na rysunku~\ref{fig:chain_real}.

\begin{figure}[htb!]
\begin{center}
\includegraphics{obrazki/schemat_real}
\makecaption{fig:chain_real}{Schemat blokowy stworzonego na potrzeby pracy toru pomiarowego będącego obiektem przeprowadzanego eksperymentu}
\end{center}
\end{figure}

Do realizacji układu wzmacniacza pomiarowego zastosowany został wzmacniacz operacyjny \enquote{MCP6002} w konfiguracji nieodwracającej, o docelowym wzmocnieniu wynoszącym~\qty{3.3}{V \per V}. Napięcie zasilania wzmacniacza wynosi~\qty{3.3}{V} i pochodzi ze stabilizatora \enquote{LD1117} typu \enquote{LDO} (ang. \enquote{Low Dropout}). Zaproponowana konfiguracja, zgodnie z wcześniejszymi założeniami, zapewnia wysoką impedancję wejściową toru pomiarowego oraz bardzo niską impedancję wyjściową wzmacniacza. Parametry statyczne oraz dynamiczne analizowanego obiektu wymagają identyfikacji, ze względu na niedostateczne informacje zawarte w dokumentacji wykorzystanego układu oraz nieznajomości dokładnego modelu dla zastosowanej aplikacji. Należy zauważyć, że dokładna analiza zastosowanego układu byłaby skomplikowana i nie jest konieczna do oceny jego podstawowych właściwości metrologicznych.

Zastosowany przetwornik analogowo-cyfrowy stanowi $12$-bitowy przetwornik wagowy~\enquote{SAR} (ang. \enquote{Successive Approximation Register}), wbudowany w mikrokontroler \enquote{STM32F411}. Źródło napięcia odniesienia stanowi w jego przypadku stabilizator \enquote{LD3985} typu \enquote{ULD} (ang. \enquote{Ultra Low Dropdown}), o znamionowym napięciu wyjściowym równym~\qty{3.3}{V}, przyłączony z użyciem dodatkowego filtru LR. Przetwornik taktowany jest wewnętrznym sygnałem zegarowym o częstotliwości~\qty{24}{MHz}, przy czym próbkowanie wyzwalane jest sygnałem z licznika, którego częstotliwość wyzwalania wynosi~\qty{48}{kHz}. Częstotliwość próbkowania analizowanego toru pomiarowego wynosi zatem $f_{s} = \qty{48}{kHz}$. Na proces konwersji analogowo-cyfrowej przypada proces kluczowania napięcia wejściowego trwający TODO taktów zegara, co odpowiada czasowi~\qty{123}{ns}, po którym następuje konwersja trwająca $15$ taktów zegara, co w analizowanym przypadku stanowi czas~\qty{625}{ns}. Łączny czas konwersji analogowo-cyfrowej pojedynczej próbki sygnału wejściowego $y(t)$ na jego dyskretną reprezentację $x(i)$ wynosi zatem~\qty{1234}{ns}. Ze względu na typową aplikacje omawianego przetwornika, jego parametry metrologiczne mogą zostać pozyskane z dokumentacji producenta układu.

Pozyskane próbki napięcia $x(i)$, stanowiące reprezentację wzmocnionego sygnału wejściowego $s(t)$, zapisywane są w wektorze $\mathbf{x}$ o długości $N = 128$. Wektor wielkości wejściowych $\mathbf{x}$ podawany jest na wejście algorytmu dyskretnej transformacji falkowej, którego implementacja zrealizowana jest z wykorzystaniem instrukcji \enquote{DSP} zastosowanego mikrokontrolera. Analizowany algorytm wykorzystuje falkę \enquote{spline4:4} dla pięciu iteracji procesu dekompozycji sygnału. Zgodnie z równaniem~\eqref{eq:alg_out_mat} wyznaczanych jest $M = 128$ próbek wektora wielkości wyjściowych $\mathbf{X}$, które stanowią wyjście pojedynczej serii pomiarowej analizowanego toru pomiarowego. Czas obliczeń wynosi w analizowanym przypadku~\qty{1508}{\micro s}, przy czym łączny czas akwizycji pojedynczego wektora próbek wielkości wejściowych wynosi~\qty{2666.6(6)}{\micro s}.

Przedstawiony tor pomiarowy pracuje w trybie ciągłym, tj. w pojedynczym oknie pomiarowym, podczas pobierania próbek wielkości wejściowych, wyznaczana jest realizacja wektora wielkości wyjściowych dla poprzedniej realizacji wektora wielkości wejściowych. Wykorzystywany jest w tym celu kontroler \enquote{DMA} (ang. \enquote{Direct Memory Access}), nadzorujący proces buforowania kolejnych realizacji sygnału $x(i)$, w czasie gdy program główny wykonuje obliczenia. Analiza wartości wielkości wyjściowych omawianego toru pomiarowego jest możliwa między innymi po podłączeniu go do komputera klasy \enquote{PC} za pośrednictwem portu \enquote{USB}, przy czym wykorzystywany jest w tym celu układ peryferyjny \enquote{USB OTG Full-Speed}, zintegrowany w zastosowanym mikrokontrolerze.

W dalszej części rozdziału zidentyfikowane zostaną najważniejsze źródła błędów analizowanego toru pomiarowego, a następnie wykorzystany zostanie zaproponowany w pracy model błędu do opisu właściwości wykazanych sygnałów błędów. Ze względu na fakt, że do budowy toru pomiarowego wykorzystano układ typu \enquote{SoC}, a zatem nie jest znany dokładny model tego toru, zostanie przedstawiona metoda identyfikacji jego właściwości istotnych ze względu na zaproponowany model błędu i jego aplikację. W ostatniej części rozdziału przeprowadzony zostanie eksperyment pomiarowy, wykorzystujący metodę Monte-Carlo, mający na celu ocenę poprawności przedstawionych rozważań i weryfikacje możliwości praktycznej aplikacji zawartych w pracy propozycji.

\section{Identyfikacja właściwości toru pomiarowego}

TODO

\section{Model błędu toru pomiarowego}

TODO

\section{Weryfikacja przedstawionych zależności}

TODO

\section{Podsumowanie przeprowadzonego eksperymentu}

TODO

