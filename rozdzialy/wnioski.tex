\chapter{Podsumowanie pracy}

Zaproponowany w pracy model błędów umożliwia opis właściwości metrologicznych torów pomiarowych złożonych zarówno z części przetwarzania analogowego, jak i cyfrowego. Model ten obejmuje dwa rodzaje właściwości: dynamiczne -- związane z widmem przetwarzanego sygnału, oraz statyczne -- niezależne od widma przetwarzanego sygnału. Stosowanie zaproponowanego modelu błędów jest możliwe bez znajomości dokładnej struktury wewnętrznej analizowanego układu, a zatem model ten znajduje swoje zastosowanie do opisu między innymi urządzeń typu \enquote{SoC}. Parametry omówionego modelu błędów mogą być pozyskiwane na drodze eksperymentu pomiarowego, jak i szacowane na podstawie dokumentacji analizowanego urządzenia.

Opis właściwości metrologicznych analizowanego obiektu wykorzystujący miarę wariancji sygnału błędu na wyjściu tego obiektu jest znacznie bardziej przystępny z punktu widzenia skomplikowania obliczeń, niż opis wykorzystujący miarę niepewności rozszerzonej (w szczególności, gdy kolejne sygnały błędów nie są ze sobą w żaden sposób skorelowane). Opis ten nie dostarcza jednak informacji o prawdopodobieństwie uzyskania danej wartości realizacji sygnału błędu, a zatem jest mniej użyteczny, niż ten wykorzystujący miarę niepewności rozszerzonej.

Zastosowana w pracy metoda szacowania wypadkowej wartości niepewności rozszerzonej, wykorzystująca metodę redukcyjnej arytmetyki interwałowej, wraz z zaproponowaną modyfikacją zapewnia wyniki zbieżne z uzyskiwanymi metodą Monte-Carlo. Ze względu na niską złożoność obliczeniową omawianej metody, jej stosowanie jest możliwe w czasie rzeczywistym, również w przypadku zmiany parametrów modelu błędów lub widma przetwarzanego sygnału. Ograniczeniem przedstawionej metody jest konieczność wstępnego wyznaczenia wartości dla współczynników kształtu, przy czym procedura ta nie może odbywać się w czasie rzeczywistym, natomiast jest ona jednorazowa. W przypadku wystąpienia korelacji pomiędzy analizowanymi sygnałami błędów należy wyznaczyć wypadkowe wartości niepewności rozszerzonej dla skorelowanych grup sygnałów lub zastosować inną, niż zaproponowano w pracy, metodę wyznaczania wartości współczynników koherencji. Niezależnie, czy obliczenia prowadzone są z wykorzystaniem budżetu niepewności obejmującego wszystkie źródła błędów, czy stosowane są wypadkowe parametry dla wyróżnionych w pracy grup sygnałów błędów, omawiana metoda zapewnia bardzo zbliżone wyniki. Należy jednak zaznaczyć, że wyznaczanie wypadkowych wartości niepewności rozszerzonej w przypadku nietypowego kształtu rozkładu wypadkowego sygnału błędu uniemożliwia wykorzystanie wyznaczonej wartości niepewności rozszerzonej w dalszych obliczeniach. Właściwość ta spowodowana jest koniecznością znajomości wartości współczynników kształtu analizowanych par sygnałów. Jako, że wartości współczynników kształtu muszą zostać wyznaczone dla wcześniej zdefiniowanych parametrów rozkładu sygnałów, wyznaczenie wartości współczynników koherencji zgodnie z proponowaną w pracy metodą jest w omawianym przypadku niemożliwe. Należy zauważyć, że w praktyce wada ta jest nieistotna, ponieważ podczas implementacji algorytmu wykorzystującego przedstawioną metodę stosowane są odpowiednio wyprowadzone zależności, wynikające z wzajemnych relacji pomiędzy kolejnymi fragmentami toru pomiarowego, co wykazano w przedstawionych przykładach.

Przedstawienie algorytmu transformacji falkowej w postaci macierzowej umożliwia analizę właściwości metrologicznych tego algorytmu w sposób zbieżny z opisywaną w pracy metodą analizy cyfrowej cześć toru pomiarowego. Omawiany algorytm przedstawić można jako zbiór cyfrowych fragmentów toru pomiarowego, których model właściwości metrologicznych opisano w pracy. Wyznaczenie wartości współczynników macierzy transformacji dla analizowanego algorytmu odbywać się może analitycznie, na podstawie założeń dotyczących stosowanej rodziny falek, lub może zostać przeprowadzone na podstawie istniejącej implementacji tego algorytmu. Dla omawianej rodziny algorytmów istnieje również możliwość wyznaczenia wartości współczynników macierzy transformacji na podstawie transmitancji kolejnych wielkości wyjściowych algorytmu, a także przeprowadzenie operacji odwrotnej do opisanej. Jako, że w klasycznym przypadku wielkości wyjściowe algorytmu można podzielić na grupy o jednakowej transmitancji, analiza może zostać uproszczona.

Jak wykazały przeprowadzone eksperymenty, dokładność oszacowania wypadkowej wartości niepewności rozszerzonej wielkości wyjściowych analizowanego toru pomiarowego uwarunkowana jest głównie dokładnością wyznaczenia parametrów dla stosowanego modelu błędów. Z uwagi na fakt, że algorytmy transformacji falkowej stanowią zbiór filtrów, istotne jest prawidłowe oszacowanie parametrów najistotniejszych źródeł błędów dla każdej grupy sygnałów błędów: statycznych, dynamicznych oraz losowych. Nawet, jeżeli wielkości wejściowe algorytmu obarczone będą pewnym sygnałem błędu dominującego, to błąd ten może cechować się widmową gęstością mocy skoncentrowaną w okolicy częstotliwości, która dla wybranej wielkości wyjściowej jest tłumiona. W takim przypadku dla analizowanej wielkości wyjściowej przenoszone będą jedynie pozostałe sygnały błędów, przy czym sygnały te mogą być dodatkowo wzmacniane. Należy zatem zauważyć, że mimo niewielkiej rozbieżności w oszacowaniu wypadkowej wartości niepewności rozszerzonej wielkości wejściowych dla opisywanych algorytmów, wypadkowa wartość niepewności rozszerzonej na wyjściu tych algorytmów może być oszacowana niewłaściwie.

Przedstawione dotychczas wnioski podsumowujące zawarte w pracy rozważania potwierdzają założenia przedstawione we wstępie do pracy oraz sformułowaną w pracy tezę. Dalsze rozważania dotyczyć mogą uzupełniania przedstawionego modelu błędów o opis kolejnych zjawisk zakłócających proces pomiaru (np. wpływu opóźnień występujących w systemach pomiarowo-sterujących na niepewność wielkości analizowanych w tych systemach). Kolejne istotne zagadnie wymagające dalszych badań stanowi rozszerzenie opisanej w pracy metody wyznaczania wartości współczynników koherencji, obejmujące przypadki sygnałów skorelowanych, umożliwiające jednolite stosowanie omawianej metody w dowolnym przypadku.

Najważniejszy walor pracy stanowi przedstawienie jednolitego, spójnego modelu błędów umożliwiającego opis parametrów sygnałów błędów obecnych w torze pomiarowym. Zaproponowany model obejmuje etapy przetwarzania analogowego, cyfrowego oraz wskazuje wzajemne relacje pomiędzy omawianymi fragmentami. Sporządzone na potrzeby pracy podsumowanie najistotniejszych informacji dotyczących metrologicznych właściwości algorytmów transformacji falkowej umożliwia zastosowanie przedstawionego modelu błędu do opisu wpływu tych algorytmów na obecne w torze pomiarowym sygnały błędów. Istotną wartość pracy stanowi również zaproponowana metoda szacowania wypadkowej wartości niepewności rozszerzonej, przy czym metoda ta zapewnia wyniki zbliżone do tych uzyskiwanych metodą Monte-Carlo, oferując jednocześnie stosunkowo niewielką, w porównaniu do pozostałych obliczeń wykonywanych przez tor pomiarowy, złożoność obliczeniową również w przypadku zmiany wartości parametrów modelu błędów. Niniejsza praca stanowi zatem propozycję, w jaki sposób opisać można właściwości metrologiczne toru pomiarowego stosującego w swojej strukturze algorytm transformacji falkowej oraz w jaki sposób ilościowo określić niedokładność wyznaczania wartości wielkości wyjściowych tego toru. Oryginalność pracy podkreśla fakt, że dotychczas nie poruszano w literaturze problemu jednolitej analizy właściwości metrologicznych algorytmów dyskretnej transformacji falkowej.
