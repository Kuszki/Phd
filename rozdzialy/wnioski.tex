\chapter{Podsumowanie pracy}

W pracy przedstawiono metodę szacowania niepewności wielkości wyjściowych torów pomiarowych wykorzystujących algorytmy transformacji falkowej. Zaproponowany w pracy model błędu umożliwiał opis właściwości metrologicznych analizowanego toru pomiarowego bez dokładnej znajomości jego modelu. Parametry konieczne do przeprowadzenia opisywanej analizy mogły zostać pozyskane eksperymentalnie lub na podstawie dokumentacji zastosowanego układu. Przedstawiona metoda ujednolica proces analizy metrologicznej, a ewentualna zmiana parametrów algorytmu transformacji falkowej nie implikuje konieczności wykonania skomplikowanych i czasochłonnych obliczeń.

Przedstawione w pracy wyniki są zbieżne zarówno z wynikami metody Monte-Carlo, jak i z wynikami eksperymentów przeprowadzonych dla rzeczywistego układu. Można zatem uznać, że przedstawiona metoda jest poprawna i może być stosowana do oceny właściwości metrologicznych torów pomiarowych wykorzystujących algorytmy transformacji falkowej.

Na podstawie pracy zauważyć można, że najbardziej korzystnym sposobem opisu parametrów błędu jest wariancja. W przypadku występowania kilku źródeł błędów, które nie są ze sobą skorelowane, metoda ta pozwala na dodawanie wariancji w celu uzyskania opisu błędu wynikowego. W przypadku opisu bazującego na odchyleniu standardowym operacje te są nieco bardziej skomplikowane. Najbardziej skomplikowanym jest opis bazujący na niepewności rozszerzonej -- w tym przypadku wyznaczanie wielkości wypadkowej wymaga znajomości kształtu rozkładu oraz relacji pomiędzy składanymi rozkładami.

Przeprowadzając analizę zgodnie z przedstawioną metodą można przyjąć różne uproszczenia, które wpływają na jej dokładność. Jeżeli na etapie analizy błędów składowych nie występuje błąd dominujący, a dodatkowo istnieje wiele źródeł błędów, można z akceptowalnym przybliżeniem założyć, że kształt rozkładu błędu będzie zbliżony do kształtu rozkładu normalnego. Jeśli natomiast błąd dominujący występuje można założyć, że kształt rozkładu błędu będzie zbliżony do kształtu rozkładu błędu dominującego.
